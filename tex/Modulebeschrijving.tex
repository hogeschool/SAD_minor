\section*{Modulebeschrijving}
\begin{tabularx}{\textwidth}{|>{\columncolor{lichtGrijs}} p{.26\textwidth}|X|}
	\hline
	\textbf{Module name:} & \modulenaam\\
	\hline
	\textbf{Module code: }& \modulecode\\
	\hline
	\textbf{Study points \newline and hours of effort:} & This module gives \stdPunten, in correspondance with 840 hours:
	\begin{itemize}
		\item 9 x 20 hours frontal lecture
		\item the rest is self-study and project work
	\end{itemize} \\
	\hline
	\textbf{Examination:} & Projects and tests (open questions and multiple choice) \\
	\hline
	\textbf{Course structure:} & Lectures, self-study, and projects \\
	\hline
	\textbf{Prerequisite knowledge:} & All programming courses, and fundamentals of security. \\
	\hline
	\textbf{Learning tools:}  &
		For the security part:
		\begin{itemize}
			\item Book: Everyday Cryptography ISBN-13: 978-0199695591
			\item Book: Cryptography Engineering ISBN-13: 978-0470474242
			\item Book: Threat Modeling Book
			\item PDF: Security Engineering
			\item Usable Security: History, Themes, and Challenges
			\item Book: The art of software security assessment ISBN-13: 978-0321444424
			\item Book: Phishing Dark Waters: The Offensive and Defensive Sides of Malicious Emails ISBN: 978-1-118-95847-6
		\end{itemize}
	
		For the formal methods part:
		\begin{itemize}
			\item Book: Friendly F\# (Fun with game programming Book 1), authors: Giuseppe Maggiore, Giulia Costantini
			\item Reader: Friendly F\# (Fun with logic programming), authors: Giuseppe Maggiore, Giulia Costantini, Francesco Di Giacomo, Gerard van Kruining
		\end{itemize} \\
	\hline
	\textbf{Connected to \newline competences:} &
		\begin{itemize}
			\item Analysis of problems with the objective of building concrete solutions
			\item Design of solution architectures in connection with a previously performed analysis
			\item Realization of solutions based on a previously performed design
			\item Advice and communication in the form of documentation and clear explanations of intentions and results
			\item Overall management of processes and team work
		\end{itemize}				
		 \\
	\hline
\end{tabularx}
\newpage

\begin{tabularx}{\textwidth}{|>{\columncolor{lichtGrijs}} p{.26\textwidth}|X|}
	\hline
	\textbf{Learning objectives:} &
	In the context of security, the student can:
	\begin{itemize}
		\item formulate strategies to adequately evaluate security \texttt{SEC1}
		\item dissect and construct the layers and objects of security \texttt{SEC2}
		\item apply and engage in the red vs blue team method either in the context of implementing/configuring or in the context of sketching/architecting \texttt{SEC3}
		\item apply and engage in the white box/laboratory study method either in the context of a code review or in the context of a user study \texttt{SEC4}
		\item apply and engage in the black box/field study method either in the context of a \textit{pentest} or in the context of a phishing campaign \texttt{SEC5}
	\end{itemize}

	In the context of formal methods, the student can:
	\begin{itemize}
		\item determine abstract complex properties \texttt{FM1}
		\item encode these complex properties into languages, tools, and other libraries \texttt{FM2}
		\item dissect the layers and objects of languages, tools, and other libraries \texttt{FM3}
		\item architect languages, tools, and other libraries in a formal specification \texttt{FM4}
		\item build languages, tools, and other libraries within a functional, logic, or declarative programming language \texttt{FM5}
	\end{itemize} \\
	\hline
	\textbf{Content:} &
	For the security part:
	\begin{itemize}
		\item Fundaments of cryptography
		\item Computer security
		\item Network security
		\item Humanoid security
	\end{itemize}
	
	For the formal methods part:
	\begin{itemize}
		\item Functional programming, with special focus on F\#
		\item Meta-programming within F\#
		\item Meta-compilers, with special focus on the school's own implementation
	\end{itemize}\\
	\hline
	\textbf{Module maintainers:} & \author\\
	\hline
	\textbf{Date:} & \today \\
	\hline
\end{tabularx}
\newpage
