\section*{Modulebeschrijving}
\begin{tabularx}{\textwidth}{|>{\columncolor{lichtGrijs}} p{.26\textwidth}|X|}
	\hline
	\textbf{Module name:} & \modulenaam\\
	\hline
	\textbf{Module code: }& \modulecode\\
	\hline
	\textbf{Study points \newline and hours of effort:} & This module gives \stdPunten, in correspondance with 840 hours:
	\begin{itemize}
		\item 9 x 20 hours frontal lecture
		\item the rest is self-study and project work
	\end{itemize} \\
	\hline
	\textbf{Examination:} & Projects and tests (open questions and multiple choice) \\
	\hline
	\textbf{Course structure:} & Lectures, self-study, and projects \\
	\hline
	\textbf{Prerequisite knowledge:} & All programming courses, and fundamentals of security. \\
	\hline
	\textbf{Learning tools:}  &
		Books relevant for possible security projects:
		\begin{itemize}
		\itemsep1pt\parskip0pt\parsep0pt
		\item
		  Human track
		\item
		  Architecting / Designing security project

		  \begin{itemize}
		  \itemsep1pt\parskip0pt\parsep0pt
		  \item
		    Threat modeling, ISBN 9781118809990
		  \item
		    Designing connected products, ISBN 9781449372569
		  \end{itemize}
		\item
		  User study project

		  \begin{itemize}
		  \itemsep1pt\parskip0pt\parsep0pt
		  \item
		    Usable security: History, themes, and challenges, ISBN 9781627055291
		  \item
		    Just enough research, ISBN 9781937557102
		  \item
		    Rocket surgery made easy, ISBN 9780321657299
		  \end{itemize}
		\item
		  Social engineering / Phishing campaign project

		  \begin{itemize}
		  \itemsep1pt\parskip0pt\parsep0pt
		  \item
		    Social engineering in IT security, ISBN 9780071818469
		  \item
		    Influence: Science and practice, ISBN 9780205609994
		  \end{itemize}
		\item
		  Technical track
		\item
		  Constructing / Building security project

		  \begin{itemize}
		  \itemsep1pt\parskip0pt\parsep0pt
		  \item
		    Secure programming HOWTO, http://www.dwheeler.com/secure-programs/
		  \item
		    Abusing the Internet of Things, ISBN 9781491902332
		  \end{itemize}
		\item
		  Code audit project

		  \begin{itemize}
		  \itemsep1pt\parskip0pt\parsep0pt
		  \item
		    The art of software security assessment, ISBN 9780321444424
		  \item
		    A bug hunter's diary, ISBN 9781593273859
		  \end{itemize}
		\item
		  Web app pentest project

		  \begin{itemize}
		  \itemsep1pt\parskip0pt\parsep0pt
		  \item
		    The Web application hacker's handbook, ISBN 9781118026472
		  \item
		    The tangled Web, ISBN 9781593273880
		  \end{itemize}
		\end{itemize}
	
		For the formal methods part:
		\begin{itemize}
			\item Book: Friendly F\# (Fun with game programming Book 1), authors: Giuseppe Maggiore, Giulia Costantini
			\item Reader: Friendly F\# (Fun with logic programming), authors: Giuseppe Maggiore, Giulia Costantini, Francesco Di Giacomo, Gerard van Kruining
		\end{itemize} \\
	\hline
	\textbf{Connected to \newline competences:} &
		Students are brought up to level 2 in the relevant competencies and to level 3 where appropriate for the software engineer. HBO-I 2014 defines security as a quality aspect of all infrastructure layers and development phases. We focus on software and infrastructure layers.\\
	\hline
\end{tabularx}
\newpage

\begin{tabularx}{\textwidth}{|>{\columncolor{lichtGrijs}} p{.26\textwidth}|X|}
	\hline
	\textbf{Learning objectives:} &
	In the context of security, the student can:
	\begin{itemize}
		\item formulate strategies to adequately evaluate security \texttt{SEC1}
		\item dissect and construct the layers and objects of security \texttt{SEC2}
		\item apply and engage in the red vs blue team method either in the context of implementing/configuring or in the context of sketching/architecting \texttt{SEC3}
		\item apply and engage in the white box/laboratory study method either in the context of a code review or in the context of a user study \texttt{SEC4}
		\item apply and engage in the black box/field study method either in the context of a \textit{pentest} or in the context of a phishing campaign \texttt{SEC5}
	\end{itemize}

	In the context of formal methods, the student can:
	\begin{itemize}
		\item determine abstract complex properties \texttt{FM1}
		\item encode these complex properties into languages, tools, and other libraries \texttt{FM2}
		\item dissect the layers and objects of languages, tools, and other libraries \texttt{FM3}
		\item architect languages, tools, and other libraries in a formal specification \texttt{FM4}
		\item build languages, tools, and other libraries within a functional, logic, or declarative programming language \texttt{FM5}
	\end{itemize} \\
	\hline
	\textbf{Content:} &
	For the security part:
	\begin{itemize}
	\itemsep1pt\parskip0pt\parsep0pt
	\item
	  Cryptography
	\end{itemize}

	\begin{enumerate}
	\def\labelenumi{\arabic{enumi}.}
	\itemsep1pt\parskip0pt\parsep0pt
	\item
	  history and applications
	\item
	  symmetric cryptography
	\item
	  asymmetric cryptography
	\item
	  key management
	\item
	  implementation issues
	\end{enumerate}

	\begin{itemize}
	\itemsep1pt\parskip0pt\parsep0pt
	\item
	  Computer security
	\end{itemize}

	\begin{enumerate}
	\def\labelenumi{\arabic{enumi}.}
	\itemsep1pt\parskip0pt\parsep0pt
	\item
	  principles of system security
	\item
	  access control and authentication
	\item
	  auditing and forensics
	\item
	  secure coding and hardening
	\item
	  reversing and exploitation
	\end{enumerate}

	\begin{itemize}
	\itemsep1pt\parskip0pt\parsep0pt
	\item
	  Network security
	\end{itemize}

	\begin{enumerate}
	\def\labelenumi{\arabic{enumi}.}
	\itemsep1pt\parskip0pt\parsep0pt
	\item
	  channels, layers, and protocols
	\item
	  architectures and perimeters
	\item
	  common networked applications
	\item
	  building distributed systems
	\item
	  prevention, detection, and response
	\end{enumerate}

	\begin{itemize}
	\itemsep1pt\parskip0pt\parsep0pt
	\item
	  Humanoid security
	\end{itemize}

	\begin{enumerate}
	\def\labelenumi{\arabic{enumi}.}
	\itemsep1pt\parskip0pt\parsep0pt
	\item
	  security management
	\item
	  social engineering and usability
	\item
	  computer crime and economics
	\item
	  international relations / security studies
	\item
	  trust, evaluation, and assurance
	\end{enumerate}
	
	For the formal methods part:
	\begin{itemize}
		\item Functional programming, with special focus on F\#
		\item Meta-programming within F\#
		\item Meta-compilers, with special focus on the school's own implementation
	\end{itemize}\\
	\hline
	\textbf{Module maintainers:} & \author\\
	\hline
	\textbf{Date:} & \today \\
	\hline
\end{tabularx}
\newpage
