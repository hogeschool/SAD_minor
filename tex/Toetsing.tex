\section{Assessment}
	The assessment of the course is divided into two main parts. Part I is a series of brainstorming sessions where groups are formed and project topics are chosen; in connection with the chosen topics, teachers will determine an appropriate set of preparation exercises that the students will need to complete after one month. Part II is one or more projects chosen by the students ranging from purely security to formal methods to hybrid solutions.


	\subsection{Grading}
		Grading of Part I is based on correctness of the assignments. The questions are either multiple choice, or are technical exercises with an either fully correct or fully incorrect answer. Each question adds one point to the total, so that the final grade each week is:
		
			$$\frac{P}{N}\times 10$$
	
		where $P$ is the number of correct answers and $N$ is the total number of answers.
		
		Grading of Part II is based on adherence to the students \textit{plan van aanpak}, and evaluation of the expected satisfaction of the client. \textbf{Handing-in} is completely based on GitHub, and the scale of grading is as follows:
		
		\begin{tabular}{| l | p{6.5cm} | p{6.5cm} |}
			\hline
			Grade range & Process & Results \\
			\hline
			0 - 2 & The students cannot document any significant work done on the project. There are few to no check-ins in the shared repository, and the repository itself is substantially empty. & There are no usable results to speak of. \\
			\hline
			2 - 4 & The students have done very little work on the project. There are few check-ins in the shared repository, and the repository itself is poorly organized and has little useful content. & A few barely usable results are visible. \\			
			\hline
			4 - 5,4 & The students have done little work on the project. There are few check-ins in the shared repository, and the repository itself has some glimpses of organization and some useful content. & Limited and barely usable results are visible. \\
			\hline
			5,5 - 7 & The students have done some work on the project. There are daily or weekly check-ins in the shared repository, and the repository itself is organized and contains useful content. & Primitive usable results are visible and clearly identified in a release accessible through the wiki pages of the repository. \\
			\hline
			7 - 10 & The students have done significant amounts of work on the project. There are daily or weekly check-ins in the shared repository, and the repository itself is organized, efficient, and contains only useful content. & Concretely and highly usable results are visible and clearly identified in a release accessible through the wiki pages of the repository. \\
			\hline
		\end{tabular} \\

		At the end of the minor all evidence must also be uploaded by students to N@tschool.
		