\section{Assessment}
	\subsection{Procedure}

	\subsection{Deliverables and deadlines}
		The assessment of the course is divided into two main parts. The first part is coupled with the background study of students, and is based on various questions and exercises. As the students are sufficiently prepared, then the second part comes into action and the students are asked to realize a series of projects.
		
		\paragraph*{Part I - background preparation}
			The background preparation will make sure that all students have the required basic knowledge to tackle the projects later on. Each week, for the first month, students will prepare on security concepts, functional programming in F\#, and the provided meta-compiler:

			\begin{itemize}
				\item \textbf{Week 1} (Friday 15:00 - 16:40) - score is 25\%
				\item \textbf{Week 2} (Monday 08:30 - 10:10) - retake of week 1

				\item \textbf{Week 2} (Friday 15:00 - 16:40) - score is 25\%
				\item \textbf{Week 3} (Monday 08:30 - 10:10) - retake of week 2

				\item \textbf{Week 3} (Friday 15:00 - 16:40) - score is 25\%
				\item \textbf{Week 4} (Monday 08:30 - 10:10) - retake of week 3

				\item \textbf{Week 4} (Friday 15:00 - 16:40) - score is 25\%
				\item \textbf{Week 5} (Monday 08:30 - 10:10) - retake of week 4

			\end{itemize}						


		\paragraph*{Part II - background preparation}
			After students have completed their background preparation, we begin with the projects. To pass the background preparation, \textit{the total score of Part I must be $\geq 4$}.

			 The projects are either based on security concepts, the application of formal methods to an actual compiler, or a mixture of the two. For a list of possible project thema's, see the appendix.

			\begin{itemize}
				\item \textbf{Week 5} (Friday - 16:00) - literature review and \textit{plan van aanpak}
				\item \textbf{Week 6} (Monday - 08:30) - reparation week 5


				\item \textbf{Week 6} (Friday - 16:00) - weekly recap
				\item \textbf{Week 7} (Monday - 08:30) - reparation week 6

				\item \textbf{Week 7} (Friday - 16:00) - weekly recap
				\item \textbf{Week 8} (Monday - 08:30) - reparation week 7

				\item \textbf{Week 8} (Friday - 16:00) - weekly recap
				\item \textbf{Week 9} (Monday - 08:30) - reparation week 8

				\item \textbf{Week 9} (Friday - 16:00) - final project report or weekly recap for longer running projects
				\item \textbf{Week 10} (Monday - 08:30) - reparation week 9
				\item \textbf{Week 10} (Friday - 08:30) - individual presentation of personal project diary through GitHub check-in comments
			\end{itemize}
		
			Weeks 10 to 15 and 15 to 20 closely follow the structure of weeks 5 to 10.
		

	\subsection{Grading}
		Grading of Part I is based on correctness of the questions. The questions are either multiple choice, or are technical exercises with an either fully correct or fully incorrect answer. Each question adds one point to the total, so that the final grade each week is:
		
			$$\frac{P}{N}\times 10$$
	
		where $P$ is the number of correct answers and $N$ is the total number of answers.
		
		Grading of Part II is based on adherence to the students plan van aanpak, and evaluation of the expected satisfaction of the client. \textbf{Handing-in} is completely based on GitHub, and the scale of grading is as follows:
		
		\begin{tabular}{| l | p{5cm} | | p{5cm}}
			\hline
			Grade range & Process & Results \\
			\hline
			0 - 2 & The students cannot document any significant work done on the project. There are few to no check-ins in the shared repository, and the repository itself is substantially empty. & There are no usable results to speak of. \\
			\hline
			2 - 4 & The students have done very little work on the project. There are few check-ins in the shared repository, and the repository itself is poorly organized and has little useful content. & A few barely usable results are visible. \\			
			\hline
			4 - 5,4 & The students have done little work on the project. There are few check-ins in the shared repository, and the repository itself has some glimpses of organization and some useful content. & Limited and barely usable results are visible. \\
			\hline
			5,5 - 7 & The students have done some work on the project. There are daily or weekly check-ins in the shared repository, and the repository itself is organized and contains useful content. & Primitive usable results are visible and clearly identified in a release accessible through the wiki pages of the repository. \\
			\hline
			7 - 10 & The students have done significant amounts of work on the project. There are daily or weekly check-ins in the shared repository, and the repository itself is organized, efficient, and contains only useful content. & Concretely and highly usable results are visible and clearly identified in a release accessible through the wiki pages of the repository. \\
		\end{tabular}		
