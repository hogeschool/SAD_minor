\section{Programma}
	The course covers a series of topics from two different branches: security and formal methods. 

	Under security we understand a variety of applied concepts related to the security of multi-user, complex systems where information has varying degrees of confidentiality.
	
	Under formal methods we understand a variety of techniques that make it possible to encode properties such as security, performance, etc. of complex systems into programming languages or similar constructs. This allows users to easily find violation of such properties reliably and without testing through advanced compilers.
	
	\subsection*{Security topics}
		\red{...........}

	\subsection*{Formal methods topics}
		The minor mainly focuses on the implementation of formal methods with a meta-compiler, provided by the lecturers. This meta-compiler is accompanied by sample implementations of programming languages, of varying degrees of complexity.
		
		The meta-compiler itself is a formal system, built in the F\# programming language. This means that as an additional level of complexity it is possible for students to add highly abstract properties to the meta-compiler instead of using it to implement languages with these properties.
