\section*{Bijlage 4: Example security project}

Name: Security evaluation

Study load: 4 ECTS / 112 hours (only indicative, you may need background reading):
12h contact, 12h homework per week, 2h presentations, 6h buffer

Format: Lectures, labs, (100 minutes hoorcollege/werkcollege) and homework.

Prerequisites: Cryptography, computer security, network security, humanoid security.

Competences: HBO-I 2014 defines security as a quality aspect of all infrastructure layers and development phases. We focus on software and infrastructure layers. Students seek level 2, and level 3 where relevant for software engineers.

Collaboration: Core Infrastructure Initiative (https://www.coreinfrastructure.org/)

Objectives: After finishing the course, the student can: independently perform design reviews, operational reviews, and a structured code audit, identify security vulnerabilities, write patches, and perform clear and responsible disclose.

Materials: The art of software security assessment, ISBN: 9780321444424 and A bug hunter’s diary, ISBN: 9781593273859

Content: Students are taught how to find security issues and when to apply audits.

Week 1: Introduction. Homework is a literature review of an analysis 	method, choosing 10 projects from the CII’s Census Project, setting up an analysis platform, making a list of project risks.

Week 2: Hand in the literature review, agree on which 10 programmes to check. Homework is design and operational reviews of the 10 open source projects, with a list of pain points and focus areas.

Week 3: Choose application to analyse (written in C, not a web app) based on code quality, documentation, compile chain, control of updates, etc. As homework, start the structured code audit.

Week 4-7: Time for individual support. Attendance required.

Week 8: Presentation of results, hand-in of report, data, and logbook.