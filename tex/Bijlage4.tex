\section*{Example security project}

Name:

Security evaluation\\\mbox{}

Study load:

4 ECTS / 112 hours (only indicative, you may need background reading): 12h contact, 12h homework per week, 2h presentations, 6h buffer\\\mbox{}

Format:

Lectures, labs, (100 minutes hoorcollege/werkcollege) and homework.\\\mbox{}

Prerequisites:

Cryptography, computer security, network security, humanoid security.\\\mbox{}

Competences:

HBO-I 2014 defines security as a quality aspect of all infrastructure layers and development phases. We focus on software and infrastructure layers. Students seek level 2, and level 3 where relevant for software engineers.\\\mbox{}

Collaboration:

Core Infrastructure Initiative (https://www.coreinfrastructure.org/)\\\mbox{}

Objectives:

After finishing the course, the student can: independently perform design reviews, operational reviews, and a structured code audit, identify security vulnerabilities, write patches, and perform clear and responsible disclose.\\\mbox{}

Materials:

The art of software security assessment, ISBN: 9780321444424

A bug hunter’s diary, ISBN: 9781593273859\\\mbox{}

Content:

Students are taught how to find security issues and when to apply audits.\\\mbox{}

Week 1:

Introduction. Homework is a literature review of an analysis 	method, choosing 10 projects from the CII’s Census Project, setting up an analysis platform, making a list of project risks.\\\mbox{}

Week 2:

Hand in the literature review, agree on which 10 programmes to check. Homework is design and operational reviews of the 10 open source projects, with a list of pain points and focus areas.\\\mbox{}

Week 3:

Choose application to analyse (written in C, not a web app) based on code quality, documentation, compile chain, control of updates, etc. As homework, start the structured code audit.\\\mbox{}

Week 4-7:

Time for individual support. Attendance required.\\\mbox{}

Week 8:

Presentation of results, hand-in of report, data, and logbook.\\\mbox{}