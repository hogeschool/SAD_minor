\section{Algemene omschrijving}
	\subsection{Inleiding}
		Functioneel programmeren is een in opkomst zijnde manier van programmeren die nogal wat verschillen vertoont met de klassiek imperatieve manier van programmeren zoals dit bekend is in programmeertalen als C, C++, Java e.v.a. \\ 

		Functionele talen zijn gebaseerd op de zogeheten \emph{$\lambda$ calculus}, waardoor deze talen fundamenteel anders werken dan de meeste programmeurs gewend zijn. Daar puur functionele talen geen side effects kennen en bovendien referential transparency gebruiken i.p.v. destructive update betekent dat programma's geschreven met functionele talen hebben hogere textit{encapsulation} van functies en data structuren, en vervolgens minder bugs en algemene fouten. Met het steeds populairder worden van deze platformen wordt ook het functioneel programmeren populairder. \\

		Verder is te zien dat met het steeds verder toenemen van het abstractieniveau, waarop computerprogramma' s functioneren de behoefte aan talen, waarin deze abstracties compact kunnen worden uitgedrukt, toeneemt. Functionele talen lijken hier tot nu toe de betere papieren voor te hebben.\\

	\subsection{Relatie met andere onderwijseenheden}
		Op deze module wordt voortgebouwd in de module TINFUN02. \\
	\subsection{Leermiddelen}
		Verplicht:
		\begin{itemize}
			\item Presentaties die gebruikt worden in de hoorcolleges (pdf): te vinden op N@tschool
			\item Opdrachten, waaraan gewerkt wordt tijdens het practicum (pdf): te vinden op N@tschool
		\end{itemize}
		Facultatief:
		\begin{itemize}
			\item Boek: Types and Programming Languages, auteur: Benjamin Pierce
			\item Boek: Friendly F\# (Fun with game programming Book 1), auteurs: Giuseppe Maggiore, Giulia Costantini
			\item Text editors: Emacs, Notepad++, Visual Studio, Xamarin Studio, etc.
		\end{itemize}